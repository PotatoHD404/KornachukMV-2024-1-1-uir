\title{Исследование и реализация сферического коня в вакууме\\
  на основе теоретико-множественного подхода}

\projecttasks{
  \projecttask{\projecttasknum}{Аналитическая часть}{}{}{}
    % (указываются предмет и цели анализа)
    \projecttask{\projectsubtasknum}
      {Изучение и сравнительный анализ … с целью…}
      {Аналитический отчет, список литературы}
      {}{}
    \projecttask{\projectsubtasknum}
      {Изучение и анализ … для…}
      {}
      {}{}
    \projecttask{\projectsubtasknum}
      {Анализ … применительно к задачам…}
      {}
      {}{}
    \projecttask{\projectsubtasknum}
      {Анализ возможностей… (для…, применительно к…, и т.п.)}
      {}
      {}{}
    \projecttask{\projectsubtasknum}
      {Оформление расширенного содержания пояснительной записки (РСПЗ)}
      {Текст РСПЗ}
      {20.10.2020}{}
  \projecttask{\projecttasknum}{Теоретическая часть}{}{}{}
    % (указываются используемые и разрабатываемые модели, методы, алгоритмы)
    \projecttask{\projectsubtasknum}
      {Используется … (модель, метод, алгоритм(ы)…) Модель/ алгоритм/метод...}
      {}
      {}{}
    \projecttask{\projectsubtasknum}
      {Выбор/разработка…}
      {}
      {}{}
    \projecttask{\projectsubtasknum}
      {Разработка…}
      {}
      {}{}
    \projecttask{\projectsubtasknum}
      {Модификация… (алгоритма, модели, и т.п.) для …}
      {}
      {}{}
    \projecttask{\projectsubtasknum}
      {Адаптация … для…}
      {}
      {}{}
  \projecttask{\projecttasknum}{Инженерная часть}{}{}{}
    % (указывается, что конкретно необходимо спроектировать, а также используемые для этого методы, технологии и инструментальные средства)
    \projecttask{\projectsubtasknum}
      {Проектирование … (системы, подсистемы, модуля…)}
      {}
      {}{}
    \projecttask{\projectsubtasknum}
      {Использовать методологию проектирования….}
      {}
      {}{}
    \projecttask{\projectsubtasknum}
      {Разработать архитектуру для… (с учетом требований к…)}
      {}
      {}{}
    \projecttask{\projectsubtasknum}
      {Результаты проектирования оформить с помощью…. При проектировании использовать язык… (например, IDEF, или UML)}
      {}
      {}{}
  \projecttask{\projecttasknum}{Технологическая и практическая часть}{}{}{}
    % (указывается, что конкретно должно быть реализовано и протестировано, а также используемые для этого методы, инструментальные средства, технологии)
    \projecttask{\projectsubtasknum}
      {Реализовать… (систему, подсистему, модуль…)}
      {Исполняемые файлы, исходный текст}
      {}{}
    \projecttask{\projectsubtasknum}
      {Протестировать… с помощью…}
      {}
      {}{}
    \projecttask{\projectsubtasknum}
      {Разработать тестовые примеры для… }
      {Исполняемые файлы, исходные тексты тестов и тестовых примеров}
      {}{}
    \projecttask{\projectsubtasknum}
      {Реализация должна иметь форму/обладать качествами...}
      {}
      {}{}
    \projecttask{\projectsubtasknum}
      {Ожидаемым результатом является программная система/программный комплекс/программное обеспечение… со следующими отличительными характеристиками…}
      {}
      {}{}
    \projecttask{\projectsubtasknum}
      {При реализации использовать технологию/платформу…}
      {}
      {}{}
  \projecttask{\projecttasknum}
    {Оформление пояснительной записки (ПЗ) и иллюстративного материала для доклада.}
    {Текст ПЗ, презентация}
    {15.12.2020}{}
}

\taskliterature{
\nocite{Sychev}
\nocite{Sokolov}
\nocite{Gaidaenko}
}

\taskdate{01.09.2020}
