\title{Апробация системы автомасштабирования кластера Kubernetes на платформах
  РУСТЭК и/или Яндекс Облако}

\taskdate{15.09.2023}

\projecttasks{
  \projecttask{\bfseries\projecttasknum}{\bfseries Аналитическая часть}{}{}{}
    % (указываются предмет и цели анализа)
    \projecttask{\projectsubtasknum}
      {Изучение и сравнительный анализ инструментов автомасштабирования
Kubernetes с целью определения лучших практик}
%
      {Аналитический отчет, список литературы}
%
      {28.09.2023}{\signat[3pt]{.2}{PAShapkin}{26.12.2023}}
      % {}{\signat[4pt]{.12}{VGPetrov}{01.01.2001}}
    \projecttask{\projectsubtasknum}
      {Изучение и анализ возможностей Terraform и Scala для реализации
автомасштабирования в Kubernetes}
%
      {}
%
      {02.10.2023}{\signat[3pt]{.2}{PAShapkin}{26.12.2023}}
    \projecttask{\projectsubtasknum}
      {Анализ интеграции Kubernetes с облачными платформами РУСТЭК и Яндекс
Облако}
%
      {}
%
      {07.10.2023}{\signat[3pt]{.2}{PAShapkin}{26.12.2023}}
    \projecttask{\projectsubtasknum}
      {Анализ возможностей оптимизации ресурсов для автомасштабирования
Kubernetes в облачных средах}
%
      {}
%
      {12.10.2023}{\signat[3pt]{.2}{PAShapkin}{26.12.2023}}
    \projecttask{\projectsubtasknum}
      {Оформление расширенного содержания пояснительной записки (РСПЗ)}
%
      {Текст РСПЗ}
%
      {22.10.2023}{\signat[3pt]{.2}{PAShapkin}{26.12.2023}}
  \projecttask{\bfseries\projecttasknum}{\bfseries Теоретическая часть}{}{}{}
    % (указываются используемые и разрабатываемые модели, методы, алгоритмы)
    \projecttask{\projectsubtasknum}
      {Разработка алгоритма автомасштабирования Kubernetes на основе анализа
производительности и доступности ресурсов}
%
      {}
%
      {27.10.2023}{\signat[3pt]{.2}{PAShapkin}{26.12.2023}}
    \projecttask{\projectsubtasknum}
      {Исследование возможностей Terraform и Scala для реализации алгоритма
автомасштабирования}
%
      {}
%
      {01.11.2023}{\signat[3pt]{.2}{PAShapkin}{26.12.2023}}
    \projecttask{\projectsubtasknum}
      {Разработка концептуальной модели данных для мониторинга и управления
ресурсами Kubernetes}
%
      {}
%
      {06.11.2023}{\signat[3pt]{.2}{PAShapkin}{26.12.2023}}
  \projecttask{\bfseries\projecttasknum}{\bfseries Инженерная часть}{}{}{}
    % (указывается, что конкретно необходимо спроектировать, а также используемые для этого методы, технологии и инструментальные средства)
    \projecttask{\projectsubtasknum}
      {Проектирование модуля автомасштабирования для Kubernetes с использованием
Terraform и Scala}
%
      {}
%
      {11.11.2023}{\signat[3pt]{.2}{PAShapkin}{26.12.2023}}
    \projecttask{\projectsubtasknum}
      {Разработка архитектуры системы, опираясь на модульность и гибкость, с
применением принципов объектно-ориентированного программирования}
%
      {}
%
      {16.11.2023}{\signat[3pt]{.2}{PAShapkin}{26.12.2023}}
    \projecttask{\projectsubtasknum}
      {Разработать детальную архитектуру модуля, включая взаимодействие с API
Kubernetes и облачными сервисами}
%
      {}
%
      {21.11.2023}{\signat[3pt]{.2}{PAShapkin}{26.12.2023}}
    \projecttask{\projectsubtasknum}
      {Оформление проектной документации с использованием UML-диаграмм для
представления архитектуры системы}
%
      {}
%
      {26.11.2023}{\signat[3pt]{.2}{PAShapkin}{26.12.2023}}
  \projecttask{\bfseries\projecttasknum}{\bfseries Технологическая и
практическая часть}{}{}{}
    % (указывается, что конкретно должно быть реализовано и протестировано, а также используемые для этого методы, инструментальные средства, технологии)
    \projecttask{\projectsubtasknum}
      {Реализация модуля автомасштабирования для Kubernetes с использованием
Terraform и Scala}
%
      {Исполняемые файлы, исходный текст}
%
      {01.12.2023}{\signat[3pt]{.2}{PAShapkin}{26.12.2023}}
    \projecttask{\projectsubtasknum}
      {Тестирование системы в различных сценариях использования на платформах
РУСТЭК и Яндекс Облако}
%
      {}
%
      {06.12.2023}{\signat[3pt]{.2}{PAShapkin}{26.12.2023}}
    \projecttask{\projectsubtasknum}
      {Разработка набора тестов для оценки эффективности и надежности системы
автомасштабирования}
%
      {Исполняемые файлы, исходные тексты тестов и тестовых примеров}
%
      {11.12.2023}{\signat[3pt]{.2}{PAShapkin}{26.12.2023}}
  \projecttask{\bfseries\projecttasknum}
    {\bfseries Оформление пояснительной записки (ПЗ) и иллюстративного материала
для доклада.}
%
    {\bfseries Текст ПЗ, презентация}
%
    {\bfseries 15.12.2023}{\signat[3pt]{.2}{PAShapkin}{26.12.2023}}
}

\taskliterature{
  \nocite{pierce-types-2012-ru}
  \nocite{wolfengagen-methods-2008}
  \nocite{wolfengagen-combinatory-2008}
  \nocite{shapkin-automation-2022}
  \nocite{bernstein-containers-2014}
  \nocite{cats}
  \nocite{cats-effect}
  \nocite{iaas2017}
  \nocite{kubectl}
  \nocite{howard2022terraform}
  \nocite{carrion2022kubernetes}
  \nocite{turin2023predicting}
  \nocite{bijon2014formal}
  \nocite{amato2018improving}
  \nocite{de2012formal}
  \nocite{bohm2021profiling}
}

% # Подписи

% Для простановки подписи используются слдеющие команды:
% - простая подпись: \sign[<сдвиг>]{<масштаб>}{<FIO>}
% - подпись с датой: \signat[<сдвиг>]{<масштаб>}{<FIO>}{<дата>}
% где 
% - <сдвиг> --- необязательный сдвиг подписи по вертикали для правильного
%   расположения относительно строки
% - <масштаб> --- число, используемое для масштарибования изображения подписи
% - <FIO> --- имя файла с подписью, файл должен быть помещен в
%   img/signatures/FIO.png и иметь прозрачный фон
% - <дата> --- дата, которая будет выведена под подписью

% ## Утверждение задания руководителем и студентом
\authortaskapproval{\signat[4pt]{.2}{MVKornachuk}{}}
\supervisortaskapproval{\signat[3pt]{.2}{PAShapkin}{}}

% ## Утверждение РСПЗ руководителем, студентом и консультантом
\authorrspzapproval{\signat[4pt]{.2}{MVKornachuk}{}}
\supervisorrspzapproval{\signat[3pt]{.2}{PAShapkin}{}}
\consultantrspzapproval{}

% ## Оценка руководителя за РСПЗ
\supervisorrspzgrade{19 из 20}

% ## Утверждение ПЗ руководителем, студентом и консультантом
\authorpzapproval{\signat[4pt]{.2}{MVKornachuk}{}}
\supervisorpzapproval{\signat[3pt]{.2}{PAShapkin}{}}
\consultantpzapproval{}

% ## Оценка руководителя за ПЗ
\supervisorpzgrade{\small(18 из 20) + (27 из 30) = \textbf{45 из 50}}
