\title{Доработка и тестирование системы автомасштабирования кластера Kubernetes}

\taskdate{15.02.2024}

\projecttasks{
      \projecttask{\bfseries\projecttasknum}{\bfseries Аналитическая
часть}{}{}{}
            \projecttask{\projectsubtasknum}
                  {Анализ результатов предыдущего этапа работы и определение
направлений для доработки и оптимизации системы автомасштабирования}
                  {Аналитический отчет}
                  {22.02.2024}{\signat[3pt]{.2}{PAShapkin}{09.04.2024}}
            \projecttask{\projectsubtasknum}
                  {Улучшения системы автомасштабирования путем улучшения
стабильности}
                  {}
                  {29.02.2024}{\signat[3pt]{.2}{PAShapkin}{09.04.2024}}
      \projecttask{\bfseries\projecttasknum}{\bfseries Теоретическая
часть}{}{}{}
                  \projecttask{\projectsubtasknum}
                  {Разработка улучшенного алгоритма автомасштабирования с учетом
новых возможностей и обновлений в используемых технологиях}
                  {}
                  {07.03.2024}{\signat[3pt]{.2}{PAShapkin}{09.04.2024}}
            \projecttask{\projectsubtasknum}
                  {Оптимизация модели данных для более эффективного мониторинга
и управления ресурсами Kubernetes}
                  {}
                  {14.03.2024}{\signat[3pt]{.2}{PAShapkin}{09.04.2024}}
      \projecttask{\bfseries\projecttasknum}{\bfseries Инженерная часть}{}{}{}
            \projecttask{\projectsubtasknum}
                  {Доработка модуля автомасштабирования с учетом улучшенного
алгоритма и оптимизированной модели данных}
                  {}
                  {21.03.2024}{\signat[3pt]{.2}{PAShapkin}{09.04.2024}}
            \projecttask{\projectsubtasknum}
                  {Рефакторинг кода для повышения производительности, читаемости
и поддерживаемости системы}
                  {}
                  {28.03.2024}{\signat[3pt]{.2}{PAShapkin}{09.04.2024}}

      \projecttask{\bfseries\projecttasknum}{\bfseries Технологическая и
практическая часть}{}{}{}
            \projecttask{\projectsubtasknum}
                  {Развертывание доработанной системы автомасштабирования в
тестовой среде Kubernetes}
                  {Исполняемые файлы, исходный текст}
                  {01.04.2024}{\signat[3pt]{.2}{PAShapkin}{09.04.2024}}

            \projecttask{\projectsubtasknum}
                  {Проведение тестирования и анализ результатов для оценки
эффективности доработанной системы}
                  {Отчет о тестировании}
                  {04.04.2024}{\signat[3pt]{.2}{PAShapkin}{09.04.2024}}
                  \projecttask{\bfseries\projecttasknum}
                  {\bfseries Оформление пояснительной записки (ПЗ) и иллюстративного материала для доклада.}%
                  {\bfseries Текст ПЗ, презентация}
                  {\bfseries 07.04.2024}{\signat[3pt]{.2}{PAShapkin}{09.04.2024}}
}

\taskliterature{
  \nocite{pierce-types-2012-ru}
  \nocite{moors2008safe}
  \nocite{cats-effect}
  \nocite{cats}
  \nocite{wolfengagen-combinatory-2008}
  \nocite{wolfengagen-methods-2008}
  \nocite{jourdan2017infrastructure}
  \nocite{morris2016infrastructure}
  \nocite{howard2022terraform}
  \nocite{shapkin-automation-2022}
  \nocite{bernstein-containers-2014}
  \nocite{sayfan2017mastering}
  \nocite{bijon2014formal}
  \nocite{kubectl}
  \nocite{carrion2022kubernetes}
  \nocite{senjab2023survey}
  \nocite{turin2023predicting}
  \nocite{davis2021bootstrapping}
  \nocite{sylla2019formal}
  \nocite{weerasiri2017taxonomy}
  \nocite{iaas2017}
  \nocite{amato2018improving}
  \nocite{de2012formal}
  \nocite{bohm2021profiling}
  \nocite{plauth2017performance}
  \nocite{nocentino2021kubernetes}
  \nocite{luksa2017kubernetes}
  \nocite{yandexcloud}
  \nocite{rustack}
  \nocite{tran2022survey}
  \nocite{qu2018auto}
  \nocite{millnert2020holoscale}
  \nocite{rodriguez2020container}
  \nocite{zhong2020cost}
  \nocite{jiang2020cloud}
  \nocite{agrawal2018log}
}

% # Подписи

% Для простановки подписи используются слдеющие команды:
% - простая подпись: \sign[<сдвиг>]{<масштаб>}{<FIO>}
% - подпись с датой: \signat[<сдвиг>]{<масштаб>}{<FIO>}{<дата>}
% где 
% - <сдвиг> --- необязательный сдвиг подписи по вертикали для правильного
%   расположения относительно строки
% - <масштаб> --- число, используемое для масштарибования изображения подписи
% - <FIO> --- имя файла с подписью, файл должен быть помещен в
%   img/signatures/FIO.png и иметь прозрачный фон
% - <дата> --- дата, которая будет выведена под подписью

% ## Утверждение задания руководителем и студентом
\authortaskapproval{\signat[4pt]{.2}{MVKornachuk}{}}
\supervisortaskapproval{\signat[3pt]{.2}{PAShapkin}{}}

% ## Утверждение РСПЗ руководителем, студентом и консультантом
\authorrspzapproval{\signat[4pt]{.2}{MVKornachuk}{}}
\supervisorrspzapproval{\signat[3pt]{.2}{PAShapkin}{}}
\consultantrspzapproval{}

% ## Оценка руководителя за РСПЗ
\supervisorrspzgrade{}

% ## Утверждение ПЗ руководителем, студентом и консультантом
\authorpzapproval{\signat[4pt]{.2}{MVKornachuk}{}}
\supervisorpzapproval{\signat[3pt]{.2}{PAShapkin}{}}
\consultantpzapproval{}

% ## Оценка руководителя за ПЗ
\supervisorpzgrade{}
