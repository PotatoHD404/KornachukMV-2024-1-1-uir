\chapter*{Реферат}
\thispagestyle{plain}

Общий объем основного текста, без учета приложений ---
\pageref{end_of_main_text} страниц, с учетом приложений ---
\pageref{end_of_document}. Количество использованных источников~---
\thetotalcitations.\\
Количество приложений~--- 5.

Ключевые слова:
\noindent
\uppercase{
terraform, scala, автомасштабирование, kubernetes, облачные платформы, Яндекс
Облако
}

Целью данной работы является доработка и тестирование системы
автомасштабирования кластера Kubernetes на платформе Яндекс Облако, с
использованием интеграции инструментов Terraform и Scala для оптимизации
управления облачной инфраструктурой.

В первой главе проводится анализ результатов предыдущего этапа работы и
определение направлений для доработки и оптимизации системы автомасштабирования.
Также исследуются возможности улучшения системы путём повышения её стабильности.

Вторая глава посвящена разработке улучшенного алгоритма автомасштабирования с
учетом новых возможностей и обновлений в используемых технологиях, а также
оптимизации модели данных для более эффективного мониторинга и управления
ресурсами Kubernetes.

Третья глава описывает доработку модуля автомасштабирования с учетом улучшенного
алгоритма и оптимизированной модели данных, а также рефакторинг кода для
повышения производительности, читаемости и поддерживаемости системы.

Четвертая глава посвящена развертыванию доработанной системы автомасштабирования
в тестовой среде Kubernetes, проведению тестирования и анализу результатов для
оценки эффективности доработанной системы.

В заключении подводятся итоги работы, формулируются основные выводы и
предложения по дальнейшему развитию и усовершенствованию системы
автомасштабирования в облачных средах.