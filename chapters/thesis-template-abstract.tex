\chapter*{Реферат}
\thispagestyle{plain}

Общий объем основного текста, без учета приложений ---
\pageref{end_of_main_text} страниц, с учетом приложений ---
\pageref{end_of_document}. Количество использованных источников~--- \thetotalcitations\  
Количество приложений~--- \thetotalappendices.

Ключевые слова:

\noindent \uppercase{
terraform,
scala,
автомасштабирование,
kubernetes,
облачные платформы,
РУСТЭК,
Яндекс Облако,
}

Целью данной работы является апробация системы автомасштабирования кластера
Kubernetes на платформах РУСТЭК и Яндекс Облако, с использованием интеграции
инструментов Terraform и Scala для оптимизации управления облачной
инфраструктурой.

В первой главе проводится анализ современных подходов к автомасштабированию в
облачных средах и инструментов развертывания контейнеризированных сред, включая
Terraform и Scala, а также исследуются системы Kubernetes и k3s.

Вторая глава посвящена разработке и реализации типизированных структур для
управления кластером Kubernetes, а также функций для взаимодействия с Terraform
и Kubernetes, включая установление биекции между этими системами.

Третья глава описывает проектирование и разработку модуля-обертки для
развертывания типизированных определений Terraform, включая описание классов и
функций для взаимодействия с Terraform и Kubernetes.

Четвертая глава посвящена реализации и экспериментальной проверке модуля,
включая разработку парсера Terraform plugins, парсера документации, модуля Case
Classes Generator и модуля для работы с Kubernetes API. Также описывается
создание абстрактного интерфейса для РУСТЭК и Яндекс Облака, а также реализация
системы автомасштабирования.

В заключении подводятся итоги работы, формулируются основные выводы и
предложения по дальнейшему развитию и усовершенствованию системы
автомасштабирования в облачных средах.

% В приложении \ref{app-format} описаны основные требования к форматированию пояснительных записок к дипломам и (магистерским) диссертациям.

% В приложении \ref{app-structure} представлена общая структура пояснительной
