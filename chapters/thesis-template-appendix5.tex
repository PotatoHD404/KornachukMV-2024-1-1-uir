\chapter{Пример использования генератора case классов}\label{sec:appendix6}

\begin{minted}{Scala}
package terraform

import io.circe.parser.decode
import io.circe.generic.auto.*
import terraform.parser.DocsParser
import terraform.parser.TerraformProviderConfig
import terraform.parser.generateCaseClasses
import java.io.File
import java.nio.file.{Files, Paths}
import scala.io.Source

def createClassFile(packageName: String,
                    classes: List[(String, String)],
                    basePath: String): Unit = {

    val packagePath = packageName.replace(".", "/")
    val fullPath = s"$basePath/$packagePath"

    val directory = new File(fullPath)
    if (!directory.exists()) {
    directory.mkdirs()
    }

    classes.foreach { case (className, classCode) =>
    val filePath = s"$fullPath/$className.scala"
    Files.write(Paths.get(filePath), classCode.getBytes)
    }
}


@main
def main(): Unit = {
    // Assuming you have the JSON string
    var source = 
    Source.fromFile("./yandex.json")
    val jsonString = source.getLines().mkString
    source.close()

    val parsedDocs = DocsParser.decodeAndFilterJson(jsonString)

    source = Source.fromFile("./yandex.json")
    val jsonStr = source.getLines().mkString
    source.close()
    val terraformProviderConfig = 
        decode[TerraformProviderConfig](jsonStr)
    terraformProviderConfig match {
    case Right(config) =>
        val generatedPackages =
         generateCaseClasses(config,
                            "terraform.providers.yandex",
                            "Yandex",
                             parsedDocs)

        val basePath = "./hcl_generator/src/main/scala"
        generatedPackages.foreach { case (packageName, classes) =>
        createClassFile(packageName, classes, basePath)
        }
    case Left(error) => println(s"Error parsing JSON: $error")
    }
}
\end{minted}
