\chapter{Разработка алгебраического описания для представления
определений Terraform}

В работе Shapkin (2022) представлен подход, основанный на использовании
аппликативных вычислительных систем и абстрактных алгебраических структур,
для автоматизации задачи построения кода развертывания и инициализации
многокомпонентных программных систем \cite{shapkin-automation-2022}.
В этом контексте, Terraform представляет собой ключевой инструмент,
позволяющий декларативно описывать инфраструктуру как код (IaC)
\cite{iaas2017}. Однако, для эффективного использования Terraform
в рамках предложенного подхода, необходимо разработать алгебраическое
описание для представления его определений.

\section{Определение основных типов и структур данных}

Введем основные типы и структуры данных, которые будут использоваться в
дальнейшем. Это важный шаг, который позволяет нам установить общий язык и
определить основные концепции, которые будут использоваться на протяжении всей
работы.

\begin{itemize}
\item $PT$ - ProviderType (тип провайдера), общий тип, содержащий в себе все
облачные сервисы. Это множество, состоящее из всех возможных провайдеров, с
которыми может работать Terraform. Математически это можно выразить как $PT =
\bigcup_{i=1}^{k}{PT_i}$, где $PT_i$ - конкретный провайдер, например $AWS$ или
$VK$ или $Yandex$. Это ключевой тип, который определяет, с какими облачными
сервисами мы можем работать.

\item $TR$ - TerraformResource, который содержит в себе все ресурсы, которые
можно создать в HCL конфиге. Это множество, состоящее из всех возможных
ресурсов, которые можно создать при помощи Terraform. Математически это можно
выразить как $TR = \bigcup_{i=1}^{k}{TR_i}$, где $TR_i$ - конкретный ресурс.
Этот тип представляет собой основу для создания конкретных ресурсов в Terraform.

\item $IR[PT]$ - InfrastructureResource (ресурс инфраструктуры) - совокупность
всех resources и datasources, которые можно создать в HCL конфиге для данного
провайдера. Это множество, которое включает в себя все возможные ресурсы и
источники данных, которые можно создать для конкретного провайдера.
Математически это можно выразить как $IR[A] = \bigcup_{i=1}^{k}{R_i[A]} \cup
\bigcup_{i=1}^{k}{D_i[A]}$, где $R_i$ - resource, $D_i$ - datasource, $A \in
PT$. Этот тип представляет собой совокупность всех ресурсов и источников данных,
которые доступны для использования в конкретном облачном сервисе.

\item $PS[PT]$ - ProviderSettings (настройки провайдера) - тип, характеризующий
настройки провайдера, которые можно задать в HCL конфиге. Это множество, которое
включает в себя все возможные настройки, которые можно задать для конкретного
провайдера. Этот тип позволяет нам управлять настройками провайдера и
адаптировать их под конкретные требования и условия использования.

\item $BR$ - BackendResource (ресурс бэкенда) - совокупность всех ресурсов,
которые можно создать в HCL конфиге для бэкенда. Это множество, которое включает
в себя все возможные ресурсы, которые можно создать для бэкенда. Математически
это можно выразить как $BR = LB \cup RB$, где $LB$ - LocalBackend, $RB$ -
RemoteBackend. Этот тип позволяет нам управлять ресурсами бэкенда и адаптировать
их под конкретные требования и условия использования.

\item $IV$, $LV$, $OV$ - InputVariable, LocalVariable, OutputVariable -
переменные, которые можно задать в HCL конфиге. Это множества, которые включают
в себя все возможные входные, локальные и выходные переменные, которые можно
использовать в конфигурации Terraform. Эти типы позволяют нам управлять
переменными, которые используются в конфигурации, и адаптировать их под
конкретные требования и условия использования.

\item $PC$ - ProviderConfig (конфиг провайдера) - тип, который содержит в себе
все возможные ресурсы, которые можно создать в HCL конфиге для данного
провайдера, а также все возможные настройки провайдера, все возможные бэкенды,
все возможные переменные. Это множество, которое включает в себя все возможные
конфигурации, которые можно создать для каждого провайдера. Математически это
можно выразить как $PC = \bigcup_i^k{PC_i[A, T1, T2, T3]}$, где $A \in PT$, $T1
\in PS[A]$, $T2 \in BR$, $T3 \in IR[A]$. Тип $PC_i$ - конкретный конфиг
провайдера, например $PC_{AWS}$ или $PC_{VK}$ или $PC_{Yandex}$. Этот тип
представляет собой комплексный конфигурационный объект, который включает в себя
все возможные ресурсы, настройки и переменные для конкретного провайдера.
\end{itemize}

Таким образом, мы ввели типы, которые позволяют описать все возможные конфиги
для всех возможных провайдеров. Эти типы и структуры данных обеспечивают основу
для нашей работы с Terraform, позволяя нам формализовать и структурировать
данные, которые мы используем.

\section{Разработка функций для обработки определений Terraform}

В контексте нашей работы с Terraform, одним из ключевых аспектов является
создание и управление определениями, которые описывают желаемую инфраструктуру.
Это включает в себя все, от определения облачных провайдеров и настроек до
ресурсов и переменных, которые используются в конфигурации. Для того чтобы
облегчить этот процесс и сделать его более управляемым, мы вводим ряд функций,
которые позволяют нам создавать эти определения и извлекать из них информацию.

\begin{itemize}
\item $f_{PC_i}[A, T_1, T_2, T_3]:$

$T_1 \times Option[T_2] \times List[T_3] \times List[IV] \times List[LV] \times
List[OV] \rightarrow PC_i[A, T_1, T_2, T_3]$, где $A \in PT$, $T_1 \in PS[A]$,
$T_2 \in BR$, $T_3 \in IR[A]$.

Это функция для создания объектов типа $PC_i[A, T_1, T_2, T_3]$. Она принимает
на вход набор параметров, которые описывают конкретный конфиг провайдера, и
возвращает объект этого типа. Эта функция играет важную роль в нашей системе,
поскольку она позволяет нам создавать конкретные конфиги провайдера на основе
заданных параметров, что обеспечивает большую гибкость и контроль над процессом
создания конфигураций.

\item $to_{HCL}$ - функция, которая преобразует объект типа $TR$ в HCL конфиг (в
его строковое представление).

$to_{HCL}: TR \rightarrow String$.

Эта функция преобразует объект типа $TR$ в его строковое представление в формате
HCL. Это ключевой элемент нашей системы, поскольку он позволяет нам генерировать
конфигурационные файлы Terraform на основе наших объектов типа $TR$. Это
обеспечивает большую гибкость и контроль над процессом создания конфигураций,
поскольку мы можем легко преобразовать наши объекты в конфигурационные файлы.

\item $to_{HCL}(PC[A, T_1, T_2, T_3])$ - функция, которая возвращает HCL конфиг
в строковом виде для типа $PC[A, T_1, T_2, T_3]$:

\begin{align*}
  to_{HCL}(PC[A, T1, T2, T3]) = \{ \\
  &allResources = Concatenate( \\
  &\quad provider, \\
  &\quad backend, \\
  &\quad resources, \\
  &\quad inputVars, \\
  &\quad localVars, \\
  &\quad outputVars \\
  &), \\
  &resourceStrings = Map(allResources, \\
  &resource \rightarrow to_{HCL}(resource)), \\
  &hcl = Join(resourceStrings, "\backslash n \backslash n") \\
  &\} \rightarrow hcl
\end{align*}

Эта функция преобразует объект типа $PC[A, T_1, T_2, T_3]$ в его строковое
представление в формате HCL. Это обеспечивает еще один уровень гибкости и
контроля над процессом создания конфигураций, поскольку мы можем легко
преобразовать наши объекты в конфигурационные файлы.

\end{itemize}

Введение этих функций является важным шагом в нашей работе с Terraform. Они
обеспечивают нам необходимые инструменты для создания и управления
конфигурациями Terraform, что в свою очередь позволяет нам более эффективно и
надежно управлять нашей инфраструктурой.

В дополнение к этим функциям, мы также используем ряд операций, которые помогают
нам обрабатывать и управлять нашими данными:

\begin{itemize}
\item $Concatenate$ - Функция, которая объединяет несколько списков в один.
Backend рассматривается как список, который может содержать ноль или один
элемент. Это позволяет нам объединять различные ресурсы и переменные в один
общий список, который затем можно использовать для генерации конфигурационного
файла Terraform.

\item $Map$ - Функция, которая применяет функцию (в данном случае, $to_{HCL}$) к
каждому элементу списка и возвращает список результатов. Это позволяет нам
преобразовать каждый ресурс или переменную в его строковое представление в
формате HCL.

\item $Join$ - Функция, которая объединяет все строки в списке в одну строку, с
разделительной строкой (в данном случае, "$\backslash$ n$\backslash$ n") между
каждым элементом. Это позволяет нам объединять все строковые представления
ресурсов и переменных в один общий конфигурационный файл Terraform.
\end{itemize}

Вместе эти функции и операции образуют мощный инструментарий для работы с
\newline
Terraform, обеспечивая нам необходимую гибкость и контроль для создания и
управления конфигурациями. Они позволяют нам
эффективно описывать и изменять нашу инфраструктуру, а также обеспечивают
надежность и безопасность нашего процесса управления инфраструктурой.


\section{Разработка функций для инкапсуляции процесса развертывания определений
Terraform}


После того как мы определили основные типы данных и функции для создания и
преобразования определений Terraform, следующим шагом является разработка
функций, которые инкапсулируют процесс развертывания этих определений. Эти
функции обеспечивают автоматизацию и упрощение процесса развертывания, позволяя
нам легко и надежно развертывать наши конфигурации Terraform.

Одной из ключевых функций, которую мы разрабатываем, является функция $deploy$,
которая принимает на вход конфигурацию провайдера $PC[A, T_1, T_2, T_3]$ и
выполняет все необходимые действия для развертывания этой конфигурации. Это
включает в себя генерацию HCL конфигурации, инициализацию Terraform, применение
конфигурации и, при необходимости, уничтожение существующей инфраструктуры.

$$
deploy: PC[A, T_1, T_2, T_3] \rightarrow State
$$

Здесь $State$ - это тип, который представляет состояние развертывания, включая
информацию о развернутых ресурсах, статусе развертывания и возможных ошибках.

Функция $deploy$ использует функцию $to_{HCL}$ для генерации HCL конфигурации из
объекта $PC[A, T_1, T_2, T_3]$. Затем она инициализирует Terraform с
использованием этой конфигурации и применяет ее, развертывая ресурсы в
соответствии с конфигурацией. Если указано, функция также может уничтожить
существующую инфраструктуру перед развертыванием новой.

Важно отметить, что функция $deploy$ обеспечивает управление состоянием
развертывания, что позволяет нам отслеживать прогресс развертывания и
обрабатывать возможные ошибки. Это обеспечивает большую надежность и
контролируемость процесса развертывания.

В дополнение к функции $deploy$, мы также разрабатываем ряд вспомогательных
функций, которые обеспечивают дополнительные возможности управления
развертыванием. Например, мы можем разработать функции для проверки статуса
развертывания, получения информации о развернутых ресурсах, или выполнения
других действий, связанных с управлением развертыванием.

\section{Выводы}

В ходе разработки алгебраического описания для представления определений
Terraform был проведен анализ основных типов и структур данных, используемых в
Terraform. Были введены типы, такие как ProviderType, TerraformResource,
InfrastructureResource, \newline ProviderSettings, BackendResource, и
ProviderConfig,
которые позволяют описать все возможные конфигурации для всех возможных
провайдеров.

Были разработаны функции для создания этих типов, а также функции для
преобразования этих типов в HCL конфигурации. Это позволяет нам автоматизировать
и упростить процесс создания конфигураций Terraform, обеспечивая большую
гибкость и контроль над этим процессом.

Также были разработаны функции для инкапсуляции процесса развертывания
определений Terraform. Эти функции позволяют нам автоматизировать и упростить
процесс развертывания, обеспечивая надежность и контролируемость этого процесса.

В целом, разработанные типы данных, функции и операции образуют мощный
инструментарий для работы с Terraform, обеспечивая необходимую гибкость и
контроль для создания и управления конфигурациями Terraform. Это позволяет нам
эффективно описывать и изменять нашу инфраструктуру, а также обеспечивает
надежность и безопасность нашего процесса управления инфраструктурой.