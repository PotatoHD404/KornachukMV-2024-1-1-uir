\chapter{Результаты проектирования \dots}

В этой главе описывается, что и как было спроектировано. При необходимости, описывается использованная методика проектирования. Сюда же относится описание внешних и внутренних программных интерфейсов, а также форматы и структуры входных и выходных данных.


\section{Использование методики <<такой-то>> для проектирования программных систем <<такого-то типа>>}

\dots

\section{Общая архитектура системы \dots}

\dots


% Команда \texorpdfstring необходима, чтобы программа просмотра PDF документов
% верно отображала текст формул в панели оглавления.
% При отсутствии команды \texorpdfstring там, где она необходима, LaTeX выводит
% предупреждение "Token not allowed in a PDF string"
\section{Архитектура подсистемы \texorpdfstring{$1$}{1}\dots}

\dots


\section{Архитектура подсистемы \texorpdfstring{$N$}{N}\dots}

\dots


\section{
  Проектирование протокола взаимодействия подсистем \texorpdfstring{$X$}{X} и
  \texorpdfstring{$Y$}{Y}
}

\dots


\section{Выводы}

Следует перечислить, какие инженерные результаты были получены, а именно: 
какие программные системы, подсистемы или модули были спроектированы. Следует 
не только назвать полученные архитектуры, но и отметить их отличительные 
особенности.
