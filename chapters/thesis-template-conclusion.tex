\chapter*{Заключение}
\addcontentsline{toc}{chapter}{Заключение}

В рамках данной работы была осуществлена доработка и
тестирование системы автомасштабирования кластера Kubernetes на платформе Яндекс
Облако. Были детально проанализированы результаты предыдущего этапа работы и
определены направления для оптимизации и улучшения системы автомасштабирования.

Особое внимание было уделено разработке улучшенного алгоритма
автомасштабирования, учитывающего новые возможности и обновления в используемых
технологиях, а также оптимизации модели данных для более эффективного
мониторинга и управления ресурсами Kubernetes.

Доработанная архитектура модуля автомасштабирования обеспечила повышение
производительности, читаемости и поддерживаемости системы за счет рефакторинга
кода и применения лучших практик разработки. Интеграция с Terraform и Scala
позволила достичь высокого уровня автоматизации и гибкости в управлении
инфраструктурой.

Развертывание доработанной системы автомасштабирования в тестовой среде
Kubernetes и проведение тестирования позволило оценить эффективность внесенных
улучшений. Результаты тестирования продемонстрировали повышение стабильности,
производительности и масштабируемости системы.

Однако, для достижения максимальной эффективности и надежности системы
автомасштабирования, необходимо продолжить работу над автоматизацией процессов
развертывания, мониторинга и управления кластером Kubernetes. Дальнейшие
исследования могут быть направлены на разработку дополнительных модулей,
отвечающих за интеллектуальный анализ данных и принятие решений о
масштабировании на основе прогнозирования нагрузки.

В целом, результаты проекта демонстрируют значительный потенциал для оптимизации
и усовершенствования процессов автомасштабирования в облачных средах.
Разработанная система, сочетающая возможности Terraform и Scala, представляет
собой эффективное решение для управления инфраструктурой и обеспечения высокой
доступности и производительности приложений в условиях динамически меняющейся
нагрузки.

Дальнейшее развитие и внедрение подобных систем автомасштабирования позволит
более эффективно использовать ресурсы облачных платформ, сократить
затраты и повысить качество предоставляемых услуг. Это особенно актуально в
свете растущей популярности микросервисной архитектуры и контейнеризации
приложений, требующих гибкого и автоматизированного управления инфраструктурой.