\chapter*{Заключение}
\addcontentsline{toc}{chapter}{Заключение}

В рамках продолжения этого учебно-исследовательского проекта была осуществлена
апробация системы автомасштабирования кластера Kubernetes на платформе
Яндекс Облако. Были детально изучены и адаптированы инструменты для
развертывания контейнеризированных сред, с акцентом на интеграцию с Terraform и
Scala. Особое внимание было уделено разработке алгебраической модели для
представления определений Terraform, включая типы данных и структуры, а также
функции для их обработки.

Разработана архитектура модуля-обертки обеспечила эффективное развертывание
типизированных определений Terraform, поддерживая генерацию HCL конфигураций и
обеспечивая взаимодействие с Kubernetes и Terraform. Реализация включала
создание парсера для плагинов Terraform, парсера документации и модуля Case
Classes Generator, а также модуля для работы с Kubernetes API.

Тестирование разработанных модулей показало их высокую эффективность и
надежность в условиях реального времени на платформе Яндекс Облако.
Однако, для достижения полной автоматизации процесса развертывания и настройки
кластера Kubernetes, требуется дополнительная разработка модуля, который будет
осуществлять изначальную развертку кластера и его настройку. Важно отметить, что
текущие модули были протестированы вручную, что подчеркивает необходимость
дальнейшей автоматизации процесса.

В целом, результаты проекта демонстрируют значительный потенциал для
автоматизации и оптимизации процессов развертывания и масштабирования приложений
в облачных средах. Для полного реализации потенциала системы и обеспечения её
надежности и удобства использования, необходимо продолжить разработку и
тестирование дополнительных модулей, особенно тех, которые отвечают за начальную
развертку и настройку кластера.
