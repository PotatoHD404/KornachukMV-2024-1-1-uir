\chapter*{Введение}
\label{sec:afterwords}
\addcontentsline{toc}{chapter}{Введение}

Эффективное управление и автомасштабирование кластеров Kubernetes в облачных
средах играют ключевую роль в современной информационной инфраструктуре.
Актуальность этой темы подчеркивается быстрым развитием технологий в области
облачных вычислений и необходимостью адаптации к постоянно меняющимся
требованиям рынка. В данном контексте, особое внимание уделяется системам, таким
как Terraform и Scala, для обеспечения надежного и гибкого управления
инфраструктурой.

Цель данной работы - апробация системы автомасштабирования кластера Kubernetes
на платформе Яндекс Облако. Основной акцент сделан на интеграции
Terraform, инструмента для управления инфраструктурой как кодом, и Scala, языка
программирования, который предоставляет мощные возможности для типизации и
модульного программирования. Эта работа направлена на создание типизированных
структур, которые обеспечат гибкость и эффективность в процессах автоматизации и
управления.


Новизна данной работы заключается в апробации системы автомасштабирования
кластера Kubernetes, сочетающей функциональные возможности Terraform и Scala для
обеспечения гибкости и масштабируемости в облачных средах.

Оригинальность исследования проявляется в разработке уникального подхода к
автомасштабированию, сочетающего эффективность Terraform в автоматизации
развертывания инфраструктуры с функциональными возможностями и статической
типизацией языка Scala.

Целью работы является апробация системы автомасштабирования кластера Kubernetes,
с акцентом на следующие задачи:

\begin{enumerate}
        \item Исследовать потенциал интеграции Terraform и Scala для
автоматизации развертывания в облачных средах.
        \item Разработать типизированные структуры для управления и
масштабирования кластера Kubernetes.
	\item Апробировать разработанные решения на платформе Яндекс Облако.
	\item Оценить эффективность и надежность предложенных подходов.
\end{enumerate}


Первая глава посвящена анализу современных тенденций в автоматизации облачных
сред и развертывания инфраструктуры, с акцентом на системы Kubernetes, Terraform
и Scala.

Вторая глава описывает процесс разработки и интеграции типизированных структур
для управления кластером Kubernetes.

Третья глава фокусируется на апробации и тестировании разработанных решений на
платформе Яндекс Облако.

Четвертая глава представляет результаты апробации и оценку эффективности
предложенных подходов.

В заключении обобщаются результаты исследования и определяются перспективы
дальнейшего развития в данной области.
