\chapter*{Введение}
\label{sec:afterwords}
\addcontentsline{toc}{chapter}{Введение}

Эффективное управление и автомасштабирование кластеров Kubernetes в облачных
средах играют ключевую роль в современной информационной инфраструктуре.
Актуальность этой темы подчеркивается быстрым развитием технологий в области
облачных вычислений и необходимостью адаптации к постоянно меняющимся
требованиям рынка. В данном контексте, особое внимание уделяется системам, таким
как Terraform и Scala, для обеспечения надежного и гибкого управления
инфраструктурой.

Цель данной работы - доработка и тестирование системы автомасштабирования
кластера Kubernetes на платформе Яндекс Облако. Основной акцент сделан на
интеграции Terraform, инструмента для управления инфраструктурой как кодом, и
Scala, языка программирования, который предоставляет мощные возможности для
типизации и модульного программирования. Эта работа направлена на оптимизацию и
усовершенствование существующих типизированных структур, которые обеспечат
гибкость, эффективность и стабильность в процессах автоматизации и управления.

Новизна данной работы заключается в доработке и оптимизации системы
автомасштабирования кластера Kubernetes, сочетающей функциональные возможности
Terraform и Scala для обеспечения улучшенной производительности,
масштабируемости и стабильности в облачных средах.

Оригинальность исследования проявляется в разработке усовершенствованного
подхода к автомасштабированию, сочетающего эффективность Terraform в
автоматизации развертывания инфраструктуры с расширенными функциональными
возможностями и статической типизацией языка Scala.

Целью работы является доработка и тестирование системы автомасштабирования
кластера Kubernetes, с акцентом на следующие задачи:

\begin{enumerate}
\item Проанализировать результаты предыдущего этапа работы и определить
направления для доработки и оптимизации системы автомасштабирования.
\item Разработать улучшенный алгоритм автомасштабирования с учетом новых
возможностей и обновлений в используемых технологиях.
\item Оптимизировать модель данных для более эффективного мониторинга и
управления ресурсами Kubernetes.
\item Провести рефакторинг кода для повышения производительности, читаемости и
поддерживаемости системы.
\item Развернуть доработанную систему автомасштабирования в тестовой среде
Kubernetes и провести тестирование для оценки эффективности улучшений.
\end{enumerate}

Первая глава посвящена анализу результатов предыдущего этапа работы и
определению направлений для доработки и оптимизации системы автомасштабирования.

Вторая глава описывает процесс разработки улучшенного алгоритма
автомасштабирования и оптимизации модели данных.

Третья глава фокусируется на доработке модуля автомасштабирования и рефакторинге
кода для повышения производительности и поддерживаемости системы.

Четвертая глава представляет результаты развертывания доработанной системы в
тестовой среде, проведения тестирования и оценки эффективности улучшений.

В заключении обобщаются результаты исследования, формулируются основные выводы и
определяются перспективы дальнейшего развития и усовершенствования системы
автомасштабирования в облачных средах.